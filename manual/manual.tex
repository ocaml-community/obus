\documentclass{article}

%% +-----------------------------------------------------------------+
%% | Packages                                                        |
%% +-----------------------------------------------------------------+

\usepackage[utf8]{inputenc}
\usepackage{xspace}
\usepackage{verbatim}
\usepackage[english]{babel}
\usepackage{amsmath}
\usepackage{amssymb}

%% +-----------------------------------------------------------------+
%% | Aliases                                                         |
%% +-----------------------------------------------------------------+

\newcommand{\obus}{\texttt{OBus}\xspace}
\newcommand{\dbus}{\texttt{D-Bus}\xspace}

%% +-----------------------------------------------------------------+
%% | Headers                                                         |
%% +-----------------------------------------------------------------+

\title{OBus user manual}
\author{Jérémie Dimino}

\begin{document}

\maketitle

\begin{abstract}
  D-Bus is an inter-processes communication protocol, or IPC for
  short, which is becomming a standard on desktop oriented
  computers. It is now possible to talk to a lot application using
  D-Bus. Moreover, it has many bindings/implementations for differents
  languages, which make it easily accessible. OBus is a pure OCaml
  implementation of it. What makes it different from other
  bindings/implementations is that it is the only one using
  cooperative threads, which make it very simple to fully exploit the
  asynchronous nature of the protocol.

  \textbf{Note:} it is advised to have some knowledge about the
  \texttt{lwt} library before reading this manual.
\end{abstract}

%% +-----------------------------------------------------------------+
%% | Table of contents                                               |
%% +-----------------------------------------------------------------+

\setcounter{tocdepth}{2}
\tableofcontents

%% +-----------------------------------------------------------------+
%% | Section                                                         |
%% +-----------------------------------------------------------------+
\section{Introduction}

OBus uses a syntax extension to ease its use, and make D-Bus method,
signal and property definitions more readable. Of course it is
possible not to use it. Code for both choices will be provided in this
manual.

\subsection{Overview of OBus}

\subsubsection{Packages}

OBus provides the folliwing packages (via findlib):

\begin{itemize}
\item ``\texttt{obus}'', which is the core library
\item ``\texttt{obus.syntax}'', which contains the syntax extension
\item ``\texttt{obus.hal}'', which is a binding to the Freedesktop
  Hal service
\item ``\texttt{obus.notification}'', which is a binding to the
  Freedesktop pop-up notifications service
\end{itemize}

\texttt{obus.hal} and \texttt{obus.notification} could be made outside
of OBus. They can be used as examples for writing bindings to D-Bus
services.

%% +-----------------------------------------------------------------+
%% | Section                                                         |
%% +-----------------------------------------------------------------+
\section{D-Bus connections, message buses}

% src/OBus_address
% src/OBus_bus
% src/OBus_connection

% src/OBus_signal
% src/OBus_property

%% +-----------------------------------------------------------------+
%% | Section                                                         |
%% +-----------------------------------------------------------------+
\section{The OBus type system}

\subsection{D-Bus types and values}

% src/OBus_value

\subsection{OBus type combinators}

% src/OBus_type
% src/OBus_pervasives

\begin{center}
  \begin{tabular}{|c|c|c|}
    \hline
    \textbf{OBus type combinator} & \textbf{D-Bus type} & \textbf{Caml type} \\
    \hline
    \texttt{char, byte} & \texttt{BYTE} & \texttt{char} \\
    \hline
    \texttt{bool, boolean} & \texttt{BOOLEAN} & \texttt{bool} \\
    \hline
    \texttt{int8} & \texttt{INT8} & \texttt{int} \\
    \hline
    \texttt{uint8} & \texttt{UINT8} & \texttt{int} \\
    \hline
    \texttt{int16} & \texttt{INT16} & \texttt{int} \\
    \hline
    \texttt{uint16} & \texttt{UINT16} & \texttt{int} \\
    \hline
    \texttt{int32} & \texttt{INT32} & \texttt{int32} \\
    \hline
    \texttt{uint32} & \texttt{UINT32} & \texttt{int32} \\
    \hline
    \texttt{int64} & \texttt{INT64} & \texttt{int64} \\
    \hline
    \texttt{uint64} & \texttt{UINT64} & \texttt{int64} \\
    \hline
    \texttt{int} & \texttt{INT32} & \texttt{int} \\
    \hline
    \texttt{uint} & \texttt{UINT32} & \texttt{int} \\
    \hline
    \texttt{float, double} & \texttt{DOUBLE} & \texttt{float} \\
    \hline
    \texttt{string} & \texttt{STRING} & \texttt{string} \\
    \hline
    \texttt{signature} & \texttt{SIGNATURE} & \texttt{OBus\_value.signature} \\
    \hline
    \texttt{path, object\_path} & \texttt{OBJECT\_PATH} & \texttt{OBus\_path.t} \\
    \hline
    \texttt{file\_descr} & \texttt{UNIX\_FD} & \texttt{Lwt\_unix.file\_descr} \\
    \hline
    \texttt{unix\_file\_descr} & \texttt{UNIX\_FD} & \texttt{Unix.file\_descr} \\
    \hline
    \texttt{$'elt$ list} & \texttt{ARRAY of $elt$} & \texttt{$'elt$ list} \\
    \hline
    \texttt{$'elt$ array} & \texttt{ARRAY of $elt$} & \texttt{$'elt$ array} \\
    \hline
    \texttt{byte\_array} & \texttt{ARRAY of byte} & \texttt{string} \\
    \hline
    \texttt{($'key$, $'value$) dict} & \texttt{ARRAY of DICT\_ENTRY of $key$ and $value$} & \texttt{($'key$ * $'value$) list} \\
    \hline
    \texttt{$'elts$ structure} & \texttt{STRUCTURE of $elts$} & \texttt{$'elts$} \\
    \hline
    \texttt{variant} & \texttt{VARIANT} & \texttt{OBus\_value.single} \\
    \hline
    \texttt{unit} & \textit{(empty sequence)} & \texttt{unit} \\
    \hline
  \end{tabular}
\end{center}

%% +-----------------------------------------------------------------+
%% | Section                                                         |
%% +-----------------------------------------------------------------+
\section{Writing bindings for D-Bus services}

% src/OBus_interface
% src/OBus_object

%% +-----------------------------------------------------------------+
%% | Section                                                         |
%% +-----------------------------------------------------------------+
\section{Tools}

\subsection{obus-dump}
\subsection{obus-introspect}
\subsection{obus-binder}

%% +-----------------------------------------------------------------+
%% | Section                                                         |
%% +-----------------------------------------------------------------+
\section{Launching a D-Bus server}

% src/OBus_server

%% +-----------------------------------------------------------------+
%% | Section                                                         |
%% +-----------------------------------------------------------------+
\section{Advanced}

% src/OBus_error

% src/OBus_info
% src/OBus_introspect
% src/OBus_message
% src/OBus_name
% src/OBus_path
% src/OBus_peer
% src/OBus_proxy
% src/OBus_resolver
% src/OBus_string
% src/OBus_uuid

% src/OBus_match

% src/OBus_wire
% src/OBus_auth
% src/OBus_transport

\end{document}
